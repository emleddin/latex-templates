\documentclass[12pt]{exam}
%% Class documentation: http://www-math.mit.edu/~psh/exam/examdoc.pdf
% Loading Things
\usepackage[margin=1.0in]{geometry}
\usepackage{enumerate} % Making lists
\usepackage[utf8]{inputenc}
\usepackage [english]{babel}
\usepackage [autostyle, english = american]{csquotes}
\MakeOuterQuote{"}
\usepackage[T1]{fontenc}    % Font encoding
\usepackage{graphicx} 		% images
\usepackage{amsmath} 		% Math
\usepackage{amssymb}					% Needed for checkboxes
% \usepackage{amssymb}		% Needed for checkboxes
% \usepackage[most]{tcolorbox} 	% Colored boxes for text with section headings
\usepackage{tcolorbox} 	 % Colored boxes for text with section headings
\usepackage[detect-all, retain-explicit-plus]{siunitx} 	% SI units
\usepackage[version=4]{mhchem} 			    % Chemical equations
\usepackage{xcolor, soul}

%% Randomize answer choices -- use with begin/end like `randomizechoices`
%% Documentation: https://ctan.math.illinois.edu/macros/latex/contrib/exam-randomizechoices/exam-randomizechoices-doc.pdf
%\usepackage{exam-randomizechoices}

%% Set random seed for the randomizer to maintain the answers' order
%% change to get different versions
%\setrandomizerseed{13}

%% Set the font
\renewcommand{\familydefault}{\sfdefault}
\usepackage{opensans}
\usepackage{sansmathfonts}
% \usepackage{helvet}

\usepackage{hyperref}

%% Set the solution box color
\definecolor{SolutionColor}{rgb}{0.8,0.9,1}

\DeclareSIUnit{\calorie}{cal}
\DeclareSIUnit{\Calorie}{\kilo\calorie}
\DeclareSIUnit{\atm}{atm}
\DeclareSIUnit{\torr}{Torr}
\DeclareSIUnit\molar{\mole\per\cubic\deci\metre}
%% Use textsc for small caps, to avoid confusion with Mega
\DeclareSIUnit\Molar{\textsc{m}}
\sisetup{per-mode=symbol}

%% Put a label in front to the question number
% \renewcommand\questionlabel{\textbf{PROBLEM \thequestion}.}

\urlstyle{same}
\hypersetup{
    linktoc=all,     %set to all if you want both sections and subsections linked
}

%% Change the footer
%\firstpagefooter{}{}{}
%\runningfooter{}{Page \thepage\ of \numpages}{}
\firstpagefooter{}{\thepage}{}
\runningfooter{}{\thepage}{}

%% Highlighting is hard, so make correct answers blue and bold
\CorrectChoiceEmphasis{\color{blue}\bfseries}

%% Define checkbox style
%\checkboxchar{$\square$}
% \checkboxchar{$\Box$}
%% Solution Answer
% \checkedchar{$\blacksquare$}

%Document Beginning
\begin{document}

\pagenumbering{arabic}

%%%%%%%%%%%%%%%%%%%%%%%%%%%%%%%%%%%%%%%%%%%%%%%%%%%%%%%%%%%%%%%%%%%%%%%%%%%%%%%
%%%%%%%%%%%%%%%%              Document Start                   %%%%%%%%%%%%%%%%
%%%%%%%%%%%%%%%%%%%%%%%%%%%%%%%%%%%%%%%%%%%%%%%%%%%%%%%%%%%%%%%%%%%%%%%%%%%%%%%

%% Use \printanswers to include the answers on the document
%% Comment it out with % to hide them
%\printanswers

%% Use \shadedsolutions to turn the outlined light gray solution box 
%% into an unoutlined dark gray shaded box
% \shadedsolutions

\noindent
\begin{table}[h]
\begin{tabular}{lr}
Student ID: \rule{1.5in}{1pt} & Name: \rule{3.25in}{1pt}
\end{tabular}
\end{table}

\begin{tcolorbox}[valign=center, box align=center, title=Instructions]
This is where you can give directions, if you want.
\end{tcolorbox}

\begin{questions}

%% Start multiple choice. Indents are for your personal benefit, not required
%% Use \ce{} to set chemical equations.
%% See the mhchem documentation for more:
%% https://mirror.csclub.uwaterloo.ca/CTAN/macros/latex/contrib/mhchem/mhchem.pdf
%% If you want to use in-line a/b/c/d/e, call \begin{oneparchoices} instead of \begin{choices}
\question Something something \ce{NO2-}. 
	\begin{choices}
	\choice A
	\CorrectChoice B
	\choice C
	\choice D
	\choice E
	\end{choices}
	
	%% Explanation
	\begin{solution} 
	Answer B is correct because I said so.
	\end{solution}

\question \label{ques:which}
What should the answer be?
    \begin{choices}
	\choice Not A
	\choice I wouldn't pick B
	\choice Why would you pick this one?
	\choice Please don't pick me
	\CorrectChoice None of these
	\end{choices}

    %% Explanation
    \begin{solution}
    Probably a trick, but "None of these" is correct.
    \end{solution}

%% I'm gonna stop with the solution block because you get the point :)

\question In question \ref{ques:which}, which was the correct choice?
    \begin{choices}
	\choice No
	\CorrectChoice E
	\choice No
	\choice No
	\choice No
	\end{choices}

\question Mark the correct formula for sugar.
    \begin{choices}
	\choice \ce{C4H5O3}
	\choice \ce{C2H5O5}
	\choice \ce{C4H10O2}
	\CorrectChoice \ce{C6H12O6}
	\choice None of these
	\end{choices}

%% Use this to insert a page break
\newpage

\question {\sethlcolor{pink}\hl{Remember to write a question here}}.
    \begin{choices}
	\choice One
	\choice Two
	\CorrectChoice Fish
	\choice Red
	\choice Blue
	\end{choices}

\question Which is the smallest?
    \begin{choices}
	\choice \SI{1}{\angstrom}
	\choice \SI{0.1}{\angstrom}
	\CorrectChoice \SI{0.001}{\angstrom}
	\choice \SI{2}{\angstrom}
	\choice \SI{48}{\angstrom}
	\end{choices}

\question Derive the following Maxwell relation: $ \displaystyle \left.\frac{\partial P}{\partial T}\right|_{V}=\left.\frac{\partial S}{\partial V}\right|_{T} $

	\begin{solution}
	
	\begin{center}
	$ dA = - S dT - P d V $
	\end{center}
	
	%% Use equation* for unnumbered equations, equation for numbered
	\begin{equation*}
	\bigg( \frac{ \partial A}{ \partial T} \bigg)_{V} = -S \quad \text{and} \quad \bigg( \frac{ \partial A}{ \partial V} \bigg)_{T} = -P
	\end{equation*}
	
	Rewrite $dA$ as:
	\begin{equation*}
	- \bigg( \frac{ \partial A}{ \partial T} \bigg)_{V} =  - \bigg( \frac{ \partial A}{ \partial V} \bigg)_{T}
	\end{equation*}
	
	\begin{equation*}
	- \bigg( \frac{\partial^{2} A}{\partial V \partial T} \bigg) = - \bigg( \frac{\partial^{2} A}{\partial T \partial V} \bigg)
	\end{equation*}
	
	\begin{equation*}
	- \bigg( \frac{\partial^{2} A}{\partial V \partial T} \bigg) = - \frac{\partial}{\partial V} \bigg( \frac{\partial A}{\partial T} \bigg)_{V} = - \bigg( \frac{\partial S}{\partial V} \bigg)_{T}
	\end{equation*}
	
	\begin{equation*}
	- \bigg( \frac{\partial^{2} A}{\partial T \partial V} \bigg) = - \frac{\partial}{\partial T} \bigg( \frac{\partial A}{\partial V} \bigg)_{T} = - \bigg( \frac{\partial P}{\partial T} \bigg)_{V}
	\end{equation*}
	
	\begin{equation*}
	- \bigg( \frac{\partial S}{\partial V} \bigg)_{T} = - \bigg( \frac{\partial P}{\partial T} \bigg)_{V} 
	\end{equation*}
	
	\begin{equation*}
	\therefore \left.\frac{\partial P}{\partial T}\right|_{V}=\left.\frac{\partial S}{\partial V}\right|_{T}
	\end{equation*}
	
	\end{solution}

% \question 
%     \begin{choices}
% 	\choice 
% 	\CorrectChoice 
% 	\choice 
% 	\choice 
% 	\choice 
% 	\end{choices}

    % %% Explanation
    % \begin{solution}
    % ABC
    % \end{solution}

% \newpage

\end{questions}


\end{document}