%%%%%%%%%%%%%%%%%%%%%%%%%%%%%%%%%%%%%%%%%%%%%%%%%%%%%%%%%%%%%%%%%
%											Preamble Start											   %
%%%%%%%%%%%%%%%%%%%%%%%%%%%%%%%%%%%%%%%%%%%%%%%%%%%%%%%%%%%%%%%%%

\documentclass[12pt]{article}
\usepackage[margin=1in]{geometry}		%% Adjusts the page set-up
\usepackage{enumerate} 				%% Allows for numbered lists
\usepackage{listings}				%% This allows for formatted computer code to be inserted
\usepackage{color}					%% Gives colors for text and coding
\usepackage{colortbl}				%% Allows table cells to be colored
\usepackage{titlesec}				%% Allows for more title and section heading options
\usepackage{textcomp}                              %% Allows upquote=true in lstlistings
\usepackage{textgreek}                              %% Greek letters
\usepackage[utf8]{inputenc}			%% Sets the unicode-8 font input encoding
\usepackage[T1]{fontenc}				%% Sets the font export stuff. Needed for proper PDF copy/paste.
\usepackage [english]{babel}			%% Sets the language
\usepackage [autostyle, english = american]{csquotes}	%% Allows you to use normal quotation marks instead of `` and ''
\usepackage{amsmath}				%% Enables American math
\usepackage{siunitx} 				%% Enables scientific notation
\usepackage[labelfont=bf]{caption} 		%% Fixes captions and makes the labels bold
\usepackage{graphicx}				%% Enables images to be inserted
\usepackage{hyperref}				%% Creates hyperlink references to specific sections/figures in PDF

%% I've defined a few colors here to insert lines of computer code
\definecolor{dkgreen}{rgb}{0,0.6,0}
\definecolor{gray}{rgb}{0.5,0.5,0.5}
\definecolor{mauve}{rgb}{0.58,0,0.82}
\definecolor{trumanpurple}{rgb}{0.32,0.04,0.52}

%% These lines define the environment for inserted coding lines. 
%% P1 is based on Python using colored words, 
%% P2 is based on Python but does not use coloration (which is good for long text strings or when open quotes are used)
\lstdefinestyle{P1}{language=python, frame=tb, aboveskip=3mm, belowskip=3mm, showstringspaces=false,
columns=flexible, basicstyle={\small\ttfamily},numbers=none, numberstyle=\tiny\color{gray},
keywordstyle=\color{blue}, commentstyle=\color{dkgreen}, stringstyle=\color{mauve},
breaklines=true, breakatwhitespace=true, tabsize=3, upquote=true}
\lstdefinestyle{P2} {language=python, frame=tb, aboveskip=3mm, belowskip=3mm, showstringspaces=false,
columns=flexible, basicstyle={\small\ttfamily}, commentstyle=\color{black},
numbers=none, breaklines=true, breakatwhitespace=true, tabsize=3, upquote=true}

\lstset{language=python,frame=tb}
\lstset{language=python,frame=tb}

%% Specifies that all images are saved in the same folder where this file is saved
\graphicspath{ { } }

%% This sets the period/spacing between the numbers and the section headings.
\titlelabel{\thetitle.\quad}

%% Without this, table captions may be right up on the table
\captionsetup[table]{skip=10pt}

%% Creates a new command for when you want to force an indent, \forceindent
\newcommand{\forceindent}{\leavevmode{\parindent=1em\indent}}

%% Change the name of figures and tables for SI (Figure S1; Table S1)
\addto\captionsenglish{\renewcommand{\thefigure}{S\arabic{figure}}}
\addto\captionsenglish{\renewcommand{\thetable}{S\arabic{table}}}

%% This means that URLs are not put into a different (monospaced) font
\urlstyle{same}

%% This sets the font as Charter instead of LaTeX's standard Computer Modern.
\usepackage{charter}
\usepackage[charter]{mathdesign}		%% Uses Charter for math font as well

%%%%%%%%%%%%%%%%%%%%%%%%%%%%%%%%%%%%%%%%%%%%%%%%%%%%%%%%%%%%%%%%%
%											Document Start											   %
%%%%%%%%%%%%%%%%%%%%%%%%%%%%%%%%%%%%%%%%%%%%%%%%%%%%%%%%%%%%%%%%%

\begin{document}
\title{\Large \bf{Supplemental Information for: \\
\large The Most Incredible of Titles}\vspace{-0.2cm}}
\author{Author M. McAuthor\vspace{-0.4cm}}
\date{\vspace{-0.5cm}}
\maketitle

Words

%%  Uncomment the following lines for a figure to appear; changing the name in the \frame line, inserting a caption, and changing the label name.
%% To reference images in the future, use something like \ref{fig:example}
%% Hint: You'll need to compile twice for the bibliography and reference labeling to work

%\begin{figure}[htbp]
%\begin{center}
%\frame{\includegraphics[height=3in]{example.png}} %%%CHANGE THE NAME HERE (and maybe the height)
%\caption{ }
%\label{fig:example}
%\end{center}
%\end{figure}

Words

%% Uncomment the following lines for a table to appear; changing the number of columns (where {|c|c|c} makes 3 columns with dividers and {c c c} has 3 columns without dividers), inserting a caption, and changing the label name.
%% To reference the table in the future, use something like \ref{table:example}
%% Columns are divided through ampersands (&), new lines are stared with (\\), and bottom separator lines are created with (\hline)
%% Units are easily added using the options of the siunitx package, and an example is shown here
%% This is what a properly formatted table looks like, but if you have an incredible amount of data/are using LaTeX on your own, 
%% I recommend using a website like https://www.tablesgenerator.com/
%% Hint: You'll need to compile twice for the bibliography and reference labeling to work

%\begin{table}[htbp]
%\centering
%\caption{\textbf{ Look at this example table caption. }}
%\label{table:example}
%\begin{tabular}{c c c}
%%\hline				%% If you wanted to have a preceding line, uncomment the beginning of this line
%\rowcolor[gray]{0.9} Column A & Column B (\si{\joule}) & Column C (\si{\meter\squared}) \\
%%\hline				%% If you wanted to have a line following this table, uncomment the beginning of this line
%A & B & C \\
%1 & 2 & 3 \\
%4 & 5 & 6 \\
%7 & 8 & 9 \\
%10 & 11 & 12 \\ 
%13 & 14 & 15
%\vspace*{-4mm}
%\end{tabular}
%\end{table} \\

In case you've ever wanted to show a random code snippet.

\begin{lstlisting}[style=P1]
print(''Hello, world!'')
\end{lstlisting}


\end{document}