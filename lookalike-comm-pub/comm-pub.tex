%%%%%%%%%%%%%%%%%%%%%%%%%%%%%%%%%%%%%%%%%%%%%%%%%%%%%%%%%%%%%%%%%
%											Preamble Start											   %
%%%%%%%%%%%%%%%%%%%%%%%%%%%%%%%%%%%%%%%%%%%%%%%%%%%%%%%%%%%%%%%%%

\documentclass[9.5pt,twocolumn]{article}
\usepackage[margin=1in]{geometry}		%% Adjusts the page set-up
\usepackage{enumerate} 				%% Allows for numbered lists
\usepackage{color}					%% Gives colors for text and coding
\usepackage{tcolorbox}				%% Gives colored tables and colors the abstract
\usepackage{colortbl}				%% Allows table cells to be colored
\usepackage{titlesec}				%% Allows for more title and section heading options
\usepackage{etoolbox}				%% This is used to left-align the title
\usepackage[utf8]{inputenc}			%% Sets the unicode-8 font input encoding
\usepackage[T1]{fontenc}				%% Sets the font export stuff. Needed for proper PDF copy/paste.
\usepackage [english]{babel}			%% Sets the language
\usepackage [autostyle, english = american]{csquotes}	%% Allows you to use normal quotation marks instead of `` and ''
\usepackage{amsmath}				%% Enables American math
%\usepackage{amssymb}				%% Enables American math symbols, cannot be used with math design package
\usepackage{siunitx} 				%% Enables scientific notation
\usepackage{setspace} 				%% Used to specify line spacing
\usepackage{empheq} 				%% Can be used to created boxed equations
\usepackage{textcomp} 				%% Allows for the use of the copyright symbol with \textcopyright
\usepackage{textgreek}                              %% Greek letters
\usepackage[labelfont=bf]{caption} 		%% Fixes captions and makes the labels bold
\usepackage{tikz} 					%% Create S Circle for supporting information (part 1)
\usepackage[superscript,biblabel]{cite} 	%% Superscripts citations
\usepackage{lettrine}					%% Used to make the first letter unnecessarily big
\usepackage{graphicx} 				%% Enables images to be inserted
\usepackage{hyperref}				%% Creates hyperlink references to specific sections/figures in PDF/hyperlinked email

%% I've defined a few colors here that make this look like an ACS journal article. You can swap jblue with TrumanPurple.
\definecolor{jbeige}{HTML}{fef3d2}
\definecolor{jhrule}{HTML}{8294cb}
\definecolor{jblue}{HTML}{0455a6}
\definecolor{TrumanPurple}{HTML}{510b76}
\definecolor{UNTgreen}{HTML}{00853E}

%% This line sets up the color box used in the abstract, including the blue lines.
\tcbset{width=\textwidth, boxrule=0pt, toprule=0.4mm, bottomrule=0.4mm, bottomtitle=0.4mm, 
colback=jbeige, colframe=jhrule, left=0pt, right=0pt, boxsep=5pt,sharp corners=all}

%% Specifies that all images are saved in the same folder where this file is saved
\graphicspath{ { } }

%% This adds a dot to the title/section numbering
\titlelabel{\thetitle.\quad}

%% These commands set the font for the section and subsection in normal text font size.
\titleformat{\section}{\normalfont\bfseries}{\thesection.}{1em}{}
\titleformat{\subsection}{\normalfont\bfseries}{\thesubsection.}{1em}{}

%% These specify how numbers appear in references
\makeatletter
\renewcommand\@biblabel[1]{(#1)}
\makeatother

%% This line renames the references section
\addto\captionsenglish{\renewcommand{\refname}{\textcolor{jblue}{$\blacksquare$ REFERENCES\vspace{-0.2cm}}}}

%% Specifies that the title, author, and affiliation will be left-aligned
\makeatletter
\patchcmd{\@maketitle}{\begin{center}}{\begin{flushleft}}{}{}
\patchcmd{\@maketitle}{\begin{tabular}[t]{c}}{\begin{tabular}[t]{@{}l}}{}{}
\patchcmd{\@maketitle}{\end{center}}{\end{flushleft}}{}{}
\makeatother

%% Creates the S Circle for supporting information (part 2)
\newcommand\encircle[1]{
\tikz[baseline=(X.base)]
   \node (X) [draw, shape=circle, inner sep=0, fill=jblue, text=white] {\strut #1};}

%% Specifies the column separation so that columns are not right on top of each other
\setlength{\columnsep}{1cm}

%% This command creates a thicker hrulefill, which are the line breaks
\def\verrulefill#1{\leaders\hrule width 0pt height#1\hfill}

%% This means that URLs are not put into a different (monospaced) font
\urlstyle{same}

%% This sets the font as Charter instead of LaTeX's standard Computer Modern.
\usepackage{charter}
\usepackage[charter]{mathdesign}		%Uses Charter for math font as well

%%%%%%%%%%%%%%%%%%%%%%%%%%%%%%%%%%%%%%%%%%%%%%%%%%%%%%%%%%%%%%%%%
%											Document Start											   %
%%%%%%%%%%%%%%%%%%%%%%%%%%%%%%%%%%%%%%%%%%%%%%%%%%%%%%%%%%%%%%%%%

\begin{document}

%% Use this title style if you have supporting information

\title{\textbf{\fontfamily{phv}{\selectfont{The Most Incredible of Titles\vspace{-0.2cm}}}}}
\author{\textsc{Author M. McAuthor\textsuperscript{*}\vspace{0.1cm}}\\
\small {Department of Chemistry, Truman State University, 100 E. Normal Street, Kirksville, MO 63501}\vspace{0.1cm} \\
\small {\encircle{S} \, Supporting Information}}
\date{\vspace{-1cm}}

%% Use this title style if do not have supporting information (uncomment this one and comment out the other)

%\title{\textbf{\fontfamily{phv}{\selectfont{The Most Incredible of Titles\vspace{-0.2cm}}}}}
%\author{\textsc{Author M. McAuthor\textsuperscript{*}\vspace{0.1cm}}\\
%\small {Department of Chemistry, Truman State University, 100 E. Normal Street, Kirksville, MO 63501}}
%\date{\vspace{-1cm}}

\twocolumn[
  \begin{@twocolumnfalse}
    \maketitle
\begin{tcolorbox}
\textcolor{jblue}{\textbf{ABSTRACT:}}
This is where an abstract goes.
\end{tcolorbox}
\vspace{2mm}
  \end{@twocolumnfalse}
  ]

%% For the lettrine command, change the {E}{} to whatever the first letter are starting your communication with. 
%% The second brackets would make that argument in uppercase superscript, which is why they are empty.
\lettrine[findent=2pt]{\textcolor{jblue}{\textbf{E}}}{} xample words for the example introduction go here. You probably get the picture of what just occurred by now. If you're not, go read some letters in the \textit{Journal of Physical Chemistry Letters}.

%% Uncomment the following lines for a figure to appear; changing the name in the \frame line, inserting a caption, and changing the label name.
%% To reference images in the future, use something like \ref{fig:example}
%% Hint: You'll need to compile twice for the bibliography and reference labeling to work

%\begin{figure}[htbp]
%\begin{center}
%\noindent\textcolor{jblue}{\hrulefill\vspace{0.2cm}}
%\frame{\includegraphics[width=\linewidth]{example.png}} %%%CHANGE THE NAME HERE
%\caption{ Look at this example figure caption. }
%\label{fig:example}
%\noindent\textcolor{jblue}{\hrulefill}
%\end{center}
%\end{figure}

More words.

%% Uncomment the following lines for a table to appear; changing the number of columns (where {|c|c|c} makes 3 columns with dividers and {c c c} has 3 columns without dividers), inserting a caption, and changing the label name.
%% To reference the table in the future, use something like \ref{table:example}
%% Columns are divided through ampersands (&), new lines are stared with (\\), and bottom separator lines are created with (\hline)
%% Units are easily added using the options of the siunitx package, and an example is shown here
%% This is what a properly formatted table looks like, but if you have an incredible amount of data/are using LaTeX on your own, 
%% I recommend using a website like https://www.tablesgenerator.com/
%% Hint: You'll need to compile twice for the bibliography and reference labeling to work

%\noindent\textcolor{jblue}{\hrulefill}
%\begin{table}[htbp]
%\centering
%\caption{\textbf{ Look at this example table caption. }}
%\label{table:example}
%\begin{tabular}{c c c}
%%\hline				%% If you wanted to have a preceding line, uncomment the beginning of this line
%\rowcolor[gray]{0.9} Column A & Column B (\si{\joule}) & Column C (\si{\meter\squared}) \\
%%\hline				%% If you wanted to have a line following this table, uncomment the beginning of this line
%A & B & C \\
%1 & 2 & 3 \\
%4 & 5 & 6 \\
%7 & 8 & 9 \\
%10 & 11 & 12 \\ 
%13 & 14 & 15
%\vspace*{-4mm}
%\end{tabular}
%\end{table} \\
%\noindent\textcolor{jblue}{\hrulefill}


More words.\cite{JournalA, JournalB}

Maybe words.\cite{Weblink}

I ran some tests and the data are these. I found stuff.

The funnel.

\textcolor{jblue}{\section*{$\blacksquare$ ASSOCIATED CONTENT}}
\subsection*{\encircle{S} \, Supporting Information}
The Supporting Information is available as an upload with this paper. \\ \indent
%% Start adding words here, if you have and SI

\textcolor{jblue}{\vspace{-0.5cm}\section*{$\blacksquare$ AUTHOR INFORMATION}}
\subsection*{Corresponding Author} \noindent
*Email: \textcolor{jblue}{\href{example@truman.edu}{example@truman.edu}}

\subsection*{Author Contributions} \noindent
The manuscript was written through contributions of all authors. / All authors have given approval to the final version of the manuscript. / $\ddagger$ These authors contributed equally.  [Match statement to author names with a symbol, as appropriate.]

\textcolor{jblue}{\vspace{-0.5cm}\subsection*{$\blacksquare$ ACKNOWLEDGEMENTS}} \noindent
Thanks to A, B, C. The author(s) acknowledge support from somewhere.

\textcolor{jblue}{\subsection*{$\blacksquare$ ABBREVIATIONS}} \noindent
Are there any abbreviations? Abbr, abbreviation; GL, good luck \\

%% Note: If you remove the abbreviations subsection, you will need to place the "\\" at the end of whatever line you end with.

%%%%%%%%%%%%%%%%%%%%%%%%%%%%%%%%%%%%%%%%%%%%%%%%%%%%%%%%%%%%%%%%%
%											References Start										   %
%%%%%%%%%%%%%%%%%%%%%%%%%%%%%%%%%%%%%%%%%%%%%%%%%%%%%%%%%%%%%%%%%

%% Note, you are limited to 99 references using this format. Hope that's not a problem.

\begin{thebibliography}{99}

\bibitem{JournalA}
Names.
Article title if you're using the long form.
\textit{Journal Name},
\textbf{Year},
\textit{Volume},
Page Numbers.

\bibitem{Weblink}
Company or agency. Title. http://examplehyperlink.com (accessed May 1, 2018).

\bibitem{JournalB}
Names.
Article title if you're using the long form.
\textit{Journal Name},
\textbf{Year},
\textit{Volume},
Page Numbers.

%% Add more as you need them.

\end{thebibliography}

\end{document}

