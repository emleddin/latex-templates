\documentclass[12pt]{article}
\usepackage[margin=1in, headheight=15pt, headsep=12pt]{geometry} %% Adjusts the page set-up
\usepackage{fancyhdr}			%% Sets up header/footer
\usepackage[table]{xcolor}		%% To define colors and color table rows
\usepackage[utf8]{inputenc}		%% Font input encoding
\usepackage [english]{babel}		%% Sets the language
\usepackage [autostyle, english = american]{csquotes} %% Allows you to use normal quotation marks instead of `` and ''
\usepackage{titlesec}			%% Allows for more title and section heading options
\usepackage{enumerate}			%% Allows for numbered lists
\usepackage{siunitx}				%% For proper SI number and units
\usepackage{setspace}			%% line spacing
\usepackage{textcomp}			%% copyright symbol and textpm
\usepackage{textgreek}			%% Greek letters
\usepackage{color,soul,colortbl}	%% To highlight what to fix, color table
\usepackage{graphicx}			%% Enables images to be inserted
\usepackage{smartdiagram}		%% For nucleic acid diagrams
\usepackage[labelfont=bf]{caption}	%% Fixes captions and makes the labels bold
\usepackage{subcaption}			%% Captions in subfigure
\usepackage{capt-of}                         %% Captions in float environment
\usepackage{hyperref}			%% Creates hyperlink references to specific sections/figures in PDF/hyperlinked email


%% Define Colors. 
%% http://latexcolor.com/ can be very helpful here
\definecolor{PalePink}{rgb}{0.98, 0.85, 0.87}
\definecolor{RedOrange}{HTML}{FFB347}
\definecolor{PaleGreen}{rgb}{0.6, 0.98, 0.6}
\definecolor{PaleBlue}{rgb}{0.67, 0.8, 0.94}
\definecolor{LtPurple}{HTML}{CA75FF}
\definecolor{Gray}{rgb}{0.81, 0.81, 0.77}
\definecolor{Mint}{HTML}{98FFCC}
\definecolor{PaleYellow}{HTML}{F5FFB1}
\definecolor{PaleOrange}{HTML}{FFCF7E}
\definecolor{PalePurple}{rgb}{0.7, 0.62, 0.71}

%% Set up subcaptions
\captionsetup[subfigure]{labelformat = parens, labelsep = space, font = small}

%% Specifies that all images are saved in the same folder where this file is saved
\graphicspath{ { } }

%% To add period after section number in TOC
\renewcommand{\thesection}{\arabic{section}.}
%% Period subsection number TOC
\renewcommand{\thesubsection}{\thesection\arabic{subsection}.}
%% Period subsubsection number TOC
\renewcommand{\thesubsubsection}{\thesubsection\arabic{subsubsection}.}


%% Set-up SI units
\DeclareSIUnit{\calorie}{cal}
\DeclareSIUnit{\kcal}{\kilo\calorie}
\sisetup{detect-all, range-units = single, range-phrase=\ \textpm\ }

%% Specify how numbers appear in references
\makeatletter
\renewcommand\@biblabel[1]{(#1)}
\makeatother

%% Change TOC name
\addto\captionsenglish{% Replace "english" with the language you use
  \renewcommand{\contentsname}%
    {Analyses}%
}

%% Set up header/footer
\pagestyle{fancy}
\fancyhf{} % clear all header and footer fields
\fancyhead[L]{Jay Doe}
\fancyhead[C]{\today}
\fancyfoot[C]{\thepage}


%Change to sans serif font
\renewcommand{\familydefault}{\sfdefault}
\everymath={\sf}

\begin{document}

%-----------------       Start Document           --------------%
\title{\large Data Report \vspace{-0.6cm}}
\author{\normalsize{Jay Doe} \vspace{-0.6cm}}
\date{\normalsize \today}

\maketitle

\tableofcontents

\pagebreak

%---------------          Systems Overview          ---------------%
\section{Overview}
\fancyhead[R]{Overview} %% Specify name of section for header

\subsection{Example Nucleic Acid}

\begin{center}
\smartdiagramset{set color list={green!40, green!40, green!40, green!40, LtPurple}, 
sequence item border color=white, sequence item font size=\footnotesize
}
\smartdiagram[sequence diagram]{T, T, T, T, \textbf{A}}
\smartdiagramset{uniform color list=yellow!40 for 5 items, 
sequence item border color=white, sequence item font size=\footnotesize,
}
\smartdiagram[sequence diagram]{A1, A2, A3, A4, \textbf{A5}}
\end{center}

\pagebreak

%---------------            OP Distances             ---------------%
\section{X-X Distances}
\fancyhead[R]{X-X Distances} %% Specify name of section for header

\begin{figure}[htbp]
    \centering
    \begin{subfigure}[b]{0.475\textwidth}   
        \centering 
        \label{fig:R1-XX}{\includegraphics[width=\linewidth]{example-image-a}}
        \caption{R1}
    \end{subfigure}
    \quad
    \begin{subfigure}[b]{0.475\textwidth}   
        \centering 
        \label{fig:R2-XX}{\includegraphics[width=\linewidth]{example-image-a}}
        \caption{R2}
    \end{subfigure}
    \vskip\baselineskip
    \begin{subfigure}[b]{0.475\textwidth}   
        \centering 
        \label{fig:R3-XX}{\includegraphics[width=\linewidth]{example-image-a}}
        \caption{R3}
    \end{subfigure}
    \caption{X-X distances for the R systems.}
\end{figure}

\begin{figure}[htbp]
    \centering
    \begin{subfigure}[b]{0.475\textwidth}
        \centering
        \label{fig:S1-XX}{\includegraphics[width=\linewidth]{example-image-a}}
        \caption{S1}
    \end{subfigure}
    \quad
    \begin{subfigure}[b]{0.475\textwidth}  
        \centering 
        \label{fig:S2-XX}{\includegraphics[width=\linewidth]{example-image-a}}
        \caption{S2}
    \end{subfigure}
    \vskip\baselineskip
    \begin{subfigure}[b]{0.475\textwidth}   
        \centering 
        \label{fig:S3-XX}{\includegraphics[width=\linewidth]{example-image-a}}
        \caption{S3}
    \end{subfigure}
    \quad
    \begin{subfigure}[b]{0.475\textwidth}   
        \centering 
        \label{fig:S4-XX}{\includegraphics[width=\linewidth]{example-image-a}}
        \caption{S4}
    \end{subfigure}
    \vskip\baselineskip
    \begin{subfigure}[b]{0.475\textwidth}   
        \centering 
        \label{fig:S5-XX}{\includegraphics[width=\linewidth]{example-image-a}}
        \caption{S5}
    \end{subfigure}
    \quad
    \begin{subfigure}[b]{0.475\textwidth}   
        \centering 
        \label{fig:S6-00}{\includegraphics[width=\linewidth]{example-image-a}}
        \caption{S6}
    \end{subfigure}
    \caption{X-X distances for the S systems.}
\end{figure}

\pagebreak

%---------------            RMSD Backbone             ---------------%
\section{Backbone RMSD}
\fancyhead[R]{Backbone RMSD} %% Specify name of section for header

\begin{figure}[htbp]
    \centering
    \begin{subfigure}[b]{0.475\textwidth}   
        \centering 
        \label{fig:R1-RMSD}{\includegraphics[width=\linewidth]{example-image-b}}
        \caption{R1}
    \end{subfigure}
    \quad
    \begin{subfigure}[b]{0.475\textwidth}   
        \centering 
        \label{fig:R2-RMSD}{\includegraphics[width=\linewidth]{example-image-b}}
        \caption{R2}
    \end{subfigure}
    \vskip\baselineskip
    \begin{subfigure}[b]{0.475\textwidth}   
        \centering 
        \label{fig:R3-RMSD}{\includegraphics[width=\linewidth]{example-image-b}}
        \caption{R3}
    \end{subfigure}
    \caption{Root mean square deviation (RMSD) for the backbone of the R systems.}
\end{figure}

\begin{figure}[htbp]
    \centering
    \begin{subfigure}[b]{0.475\textwidth}
        \centering
        \label{fig:S1-RMSD}{\includegraphics[width=\linewidth]{example-image-b}}
        \caption{S1}
    \end{subfigure}
    \quad
    \begin{subfigure}[b]{0.475\textwidth}  
        \centering 
        \label{fig:S2-RMSD}{\includegraphics[width=\linewidth]{example-image-b}}
        \caption{S2}
    \end{subfigure}
    \vskip\baselineskip
    \begin{subfigure}[b]{0.475\textwidth}   
        \centering 
        \label{fig:S3-RMSD}{\includegraphics[width=\linewidth]{example-image-b}}
        \caption{S3}
    \end{subfigure}
    \quad
    \begin{subfigure}[b]{0.475\textwidth}   
        \centering 
        \label{fig:S4-RMSD}{\includegraphics[width=\linewidth]{example-image-b}}
        \caption{S4}
    \end{subfigure}
    \vskip\baselineskip
    \begin{subfigure}[b]{0.475\textwidth}   
        \centering 
        \label{fig:S5-RMSD}{\includegraphics[width=\linewidth]{example-image-b}}
        \caption{S5}
    \end{subfigure}
    \quad
    \begin{subfigure}[b]{0.475\textwidth}   
        \centering 
        \label{fig:S6-RMSD}{\includegraphics[width=\linewidth]{example-image-b}}
        \caption{S6}
    \end{subfigure}
    \caption{Root mean square deviation (RMSD) for the backbone of the S systems.}
\end{figure}

\pagebreak

%---------------             RMSF             ---------------%
\section{Root Mean Square Fluctuations}
\fancyhead[R]{RMSF} %% Specify name of section for header

\begin{figure}[htbp]
    \centering
    \begin{subfigure}[b]{0.475\textwidth}   
        \centering 
        \label{fig:R1-RMSF}{\includegraphics[width=\linewidth]{example-image-c}}
        \caption{R1}
    \end{subfigure}
    \quad
    \begin{subfigure}[b]{0.475\textwidth}   
        \centering 
        \label{fig:R2-RMSF}{\includegraphics[width=\linewidth]{example-image-c}}
        \caption{R2}
    \end{subfigure}
    \vskip\baselineskip
    \begin{subfigure}[b]{0.475\textwidth}   
        \centering 
        \label{fig:R3-RMSF}{\includegraphics[width=\linewidth]{example-image-c}}
        \caption{R3}
    \end{subfigure}
    \caption{Root mean square fluctuation (RMSF) for the backbone of the R systems.}
\end{figure}

\begin{figure}[htbp]
    \centering
    \begin{subfigure}[b]{0.475\textwidth}
        \centering
        \label{fig:S1-RMSF}{\includegraphics[width=\linewidth]{example-image-c}}
        \caption{S1}
    \end{subfigure}
    \quad
    \begin{subfigure}[b]{0.475\textwidth}  
        \centering 
        \label{fig:S2-RMSF}{\includegraphics[width=\linewidth]{example-image-c}}
        \caption{S2}
    \end{subfigure}
    \vskip\baselineskip
    \begin{subfigure}[b]{0.475\textwidth}   
        \centering 
        \label{fig:S3-RMSF}{\includegraphics[width=\linewidth]{example-image-c}}
        \caption{S3}
    \end{subfigure}
    \quad
    \begin{subfigure}[b]{0.475\textwidth}   
        \centering 
        \label{fig:S4-RMSF}{\includegraphics[width=\linewidth]{example-image-c}}
        \caption{S4}
    \end{subfigure}
    \vskip\baselineskip
    \begin{subfigure}[b]{0.475\textwidth}   
        \centering 
        \label{fig:S5-RMSF}{\includegraphics[width=\linewidth]{example-image-c}}
        \caption{S5}
    \end{subfigure}
    \quad
    \begin{subfigure}[b]{0.475\textwidth}   
        \centering 
        \label{fig:S6-RMSF}{\includegraphics[width=\linewidth]{example-image-c}}
        \caption{S6}
    \end{subfigure}
    \caption{Root mean square fluctuation (RMSF) for the backbone of the S systems.}
\end{figure}

\pagebreak

%---------------             HBA             ---------------%
\section{Hydrogen Bond Analysis}
\fancyhead[R]{HBA} %% Specify name of section for header

%% BEGIN TABLE ---------------------------------------------------
\begin{table}[htb!]
    \caption{Selected values for hydrogen bond interactions across total simulation time. All values are an average of the total simulation time for each replicate (\si{\percent}). Acceptor--donor interactions affected at a \SI{20}{\percent} threshold are highlighted yellow, at a \SI{50}{\percent} threshold are highlighted orange, and at an \SI{80}{\percent} threshold are highlighted purple.}
    \label{tab:HBA}
    \centering
    \begin{tabular}{|c|c|r|r|r|r|r|r|}
    \hline
Acceptor    & Donor     & WT A & A123B & A123C & A123D & A123E & A123F\\ \hline
% \rowcolor{PaleOrange}
% \rowcolor{PaleYellow}
% \rowcolor{PalePurple}
% & 0.00  & 0.00  & 0.00 & 0.00    & 0.00  & 0.00 \\
%
\rowcolor{PaleOrange}
C112@O	& S122@N	& 1.00  & 60.00 	& 61.00 	& -     	& 62.00	& 63.00 \\
%
\rowcolor{PalePurple}
R543@O	& R532@NH2	& 1.00  & -		& - 		& 82.00	& 83.00	& 85.00 \\
% 
\rowcolor{PaleYellow}
G302@O	& R938@N1	& 80.00  & 45.00	& 47.00	& 48.00	& 49.00	& 50.00 \\ \hline
    \end{tabular}
\end{table}
%% END TABLE ------------------------------------------------

% \pagebreak

%%---------------            Normal Mode Analysis             ---------------%
%\section{Normal Mode Analysis}
%\fancyhead[R]{NMA} %% Specify name of section for header

% \pagebreak

%%---------------       Energy Decomposition Analysis        ---------------%
%\section{Energy Decomposition Analysis}
%\fancyhead[R]{EDA} %% Specify name of section for header

% \pagebreak

%%---------------       Dynamic Cross-Correlation        ---------------%
%\section{Dynamic Cross-Correlation}
%\fancyhead[R]{MC} %% Specify name of section for header

% \pagebreak

%%---------------           MC Differences            ---------------%
%\section{Differences in Dynamic Cross-Correlation}
%\fancyhead[R]{MC Differences} %% Specify name of section for header

% \pagebreak

\end{document}